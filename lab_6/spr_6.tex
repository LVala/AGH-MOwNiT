\documentclass{article}

\usepackage{polski}
\usepackage{amsmath}
\usepackage{graphicx}
\usepackage{float}
\usepackage{subfig}
\usepackage{multirow}

\title{Rozwiązywanie równań i układów równań nieliniowych}
\author{\textbf{Łukasz Wala}\\
    \textit{AGH, Wydział Informatyki, Elektroniki i Telekomunikacji} \\
    \textit{Metody Obliczeniowe w Nauce i Technice 2021/2022}}
\date{Kraków, \today}

\begin{document}
\maketitle

\section{Problem 1}
\subsection{Opis problemu}
Dany jest układ równań liniowych \textbf{Ax}=\textbf{b}.
Elementy macierzy \textbf{A} o wymiarze $n$\,x\,$n$ są określone wzorem:
$$
\begin{cases}
    a_{1j}=1\\
    a_{ij}=\frac{1}{i+j-1} \ dla \ i \ne j
\end{cases}i,j=1,...,n
$$
Za wektor \textbf{x} przyjęta zostanie dowolna $n$-elementowa permutacja ze zbioru \{1,-1\} i obliczony zostanie
wektor \textbf{b}. Następnie metodą eliminacji Gaussa rozwiązany zostanie układ równań liniowych \textbf{Ax}=\textbf{b}
(przyjmując jako niewiadomą wektor \textbf{x}). Przyjęte zostaną różne precyjze dla znanych wartości macierzy \textbf{A} i
wektora \textbf{b}. Sprawdzone zostanie to, jak błędy zaokrągleń zaburzają rozwiązanie dla różnych rozmiarów układu
(porównany \textbf{x} obliczony z \textbf{x} zadanym). Eksperyment przeprowadzony zostanie dla różnych rozmiarów układu.

\subsection{Opracowanie problemu}
Program użyty do rozwiązania układu został napisany w języku Python z użyciem pakiety numpy.

Zakres użytego $n$ wynosi 3-100, natomiast, w celu uzyskania różnych precyzji, zostaną użyte typy float128, float64 oraz float32
(128 bitów, 64 bity oraz 32 bitów, typy zmiennoprzecinkowe). Przyjęty wektor \textbf{x} składa się naprzemiennie z 1 oraz -1, czyli
\textbf{x}$=[1,-1,1,-1,...]$.

Poniżej wyniki dla wartości 3-18 z użyciem float64 $n$:
\begin{table}[H]
\centering
\begin{tabular}{|l|p{11cm}|}
\hline
$n$ & Wynik (float64)\\ \hline
3 & [ 1. -1.  1.] \\ \hline
4 & [ 1. -1.  1. -1.] \\ \hline
5 & [ 1. -1.  1. -1.  1.] \\ \hline
6 & [ 1. -1.  1. -1.  1. -1.] \\ \hline
7 & [ 1.     -1.      1.     -0.99999999  0.99999999 -0.99999999  1. ] \\ \hline
8 & [ 1.         -1.          1.00000001 -1.00000003  1.00000007 -1.00000008 1.00000005 -1.00000001] \\ \hline
9 & [ 1.         -1.          1.00000002 -1.00000008  1.00000022 -1.00000034 1.00000031 -1.00000015  1.00000003] \\ \hline
10 & [ 1.         -0.99999992  0.99999887 -0.999992    0.99996863 -0.99992652 0.99989471 -0.99990954  0.9999572  -0.99999143] \\ \hline
11 & [ 0.99999997 -0.99999848  0.99997297 -0.99976817  0.99887436 -0.99664975 0.99366282 -0.99235204  0.99429537 -0.99760317  0.99956614] \\ \hline
12 & [ 0.99999941 -0.99996221  0.99919827 -0.99172934  0.95111984 -0.81988143 0.56753005 -0.31190968  0.28013607 -0.52390499  0.81951835 -0.97011435] \\ \hline
13 & [  0.99999778  -0.99983381   0.99588177  -0.95002628   0.6492147
0.55558381  -3.57758546   8.16116812 -11.53222833  10.53875693
-5.83889397   1.35691714   0.6410476 ] \\ \hline
14 & [  0.99999839  -0.99987809   0.99693457  -0.96208463   0.72726558
0.24775735  -2.82147498   7.05683998 -10.81750725  10.99258477
-7.21472848   2.59779619   0.10075086  -0.90425428] \\ \hline
15 & [ 1.00000032 -1.00002608  1.00068109 -1.0084821   1.05859506 -1.23642266
1.52425818 -1.31770342 -0.63773969  4.60311654 -8.06911186  7.84695378
-4.30073459  0.80508738  0.73152805] \\ \hline
16 & [  1.00000035  -1.00002873   1.00077369  -1.0099144    1.07044155
-1.2924074    1.66629056  -1.39635283  -1.36182391   7.41476733
-13.5730076   14.63721333  -9.78616429   3.63410525  -0.11663269
-0.88726018] \\ \hline
17 & [  0.99999998  -1.00000181   1.00012819  -1.00256848   1.02437038
-1.12426363   1.32037494  -1.09632243  -1.12225943   6.55223303
-12.60591686  13.4139281   -7.23211203  -0.09782497   2.92766818
-2.18587769   1.22844453] \\ \hline
18 & [  1.00000009  -1.00000875   1.00027248  -1.00388918   1.029806
  -1.12729424   1.25609572  -0.81278282  -1.66686582   6.95901598
 -12.29735709  12.50041866  -6.48195112  -0.38410709   3.12164386
  -2.45497509   1.40118285  -1.03920443] \\ \hline
\end{tabular}
\caption{Wyniki dla wartości 3-18 $n$}
\end{table}

Już dla $n=12$ widać istotne błędy, które wraz ze wzrastającym $b$ będą stawały się coraz bardziej poważne.
Poniżej tabele błędów dla wszystkich badanych wartości i precyzji (błąd - norma euklidesowa wektora prawdziwych wyników i tego
uzyskanego za pomocą eliminacji Gaussa):

\newpage
\thispagestyle{empty}

\begin{table}[H]
\parbox{.45\linewidth}{
\hspace*{-2.4cm}
\begin{tabular}{|l|l|l|l|}
\hline
$n$ & float128 & float64 & float32 \\ \hline
3 & 0.00000e+00 & 0.00000e+00 & 0.00000e+00 \\ \hline
4 & 6.64652e-15 & 3.01871e-13 & 0.00000e+00 \\ \hline
5 & 7.28264e-13 & 9.22938e-12 & 4.06851e-12 \\ \hline
6 & 5.44100e-11 & 3.63798e-10 & 1.08591e-11 \\ \hline
7 & 3.69517e-09 & 1.36093e-08 & 1.80863e-09 \\ \hline
8 & 6.74760e-09 & 1.20335e-07 & 6.03675e-09 \\ \hline
9 & 4.03806e-08 & 5.40051e-07 & 1.62591e-06 \\ \hline
10 & 6.73052e-09 & 1.66203e-04 & 2.23585e-05 \\ \hline
11 & 8.55678e-09 & 1.22341e-02 & 1.51898e-03 \\ \hline
12 & 1.10052e-08 & 1.21398e+00 & 5.23169e-02 \\ \hline
13 & 2.39652e-08 & 2.12151e+01 & 5.62548e-01 \\ \hline
14 & 1.81406e-08 & 2.11156e+01 & 3.72763e-01 \\ \hline
15 & 2.39101e-08 & 1.50485e+01 & 6.50496e-01 \\ \hline
16 & 3.24268e-08 & 2.59432e+01 & 1.17698e+00 \\ \hline
17 & 7.13559e-08 & 2.29849e+01 & 2.33258e+00 \\ \hline
18 & 2.67034e-08 & 2.21990e+01 & 1.50270e+00 \\ \hline
19 & 3.79751e-08 & 8.53660e+01 & 2.24009e+01 \\ \hline
20 & 1.77932e-08 & 8.71401e+02 & 8.78090e+00 \\ \hline
21 & 2.25166e-08 & 5.86659e+01 & 8.24051e+00 \\ \hline
22 & 1.56222e-08 & 5.29518e+01 & 4.86432e+00 \\ \hline
23 & 4.74004e-08 & 4.70098e+01 & 4.75505e+00 \\ \hline
24 & 1.85685e-07 & 1.23774e+02 & 9.36016e+00 \\ \hline
25 & 1.83021e-07 & 2.69255e+02 & 1.50937e+01 \\ \hline
26 & 1.24864e-07 & 1.11393e+02 & 1.35080e+01 \\ \hline
27 & 1.23479e-07 & 1.24288e+02 & 3.16106e+01 \\ \hline
28 & 1.80821e-06 & 1.57776e+03 & 7.67399e+01 \\ \hline
29 & 9.48392e-08 & 4.02865e+02 & 2.22652e+01 \\ \hline
30 & 1.57316e-08 & 1.89246e+02 & 6.82841e+00 \\ \hline
31 & 1.17393e-08 & 2.24508e+02 & 3.83671e+00 \\ \hline
32 & 1.48993e-08 & 6.72210e+01 & 6.23303e+00 \\ \hline
33 & 4.08596e-08 & 7.39117e+01 & 1.14283e+01 \\ \hline
34 & 5.13099e-08 & 5.53441e+02 & 4.80933e+01 \\ \hline
35 & 2.78104e-08 & 5.19503e+02 & 4.10183e+01 \\ \hline
36 & 4.51019e-08 & 3.53450e+02 & 3.31293e+01 \\ \hline
37 & 2.31974e-08 & 3.91273e+02 & 1.13820e+01 \\ \hline
38 & 5.33618e-07 & 3.77185e+02 & 2.47144e+01 \\ \hline
39 & 4.03595e-08 & 1.87523e+02 & 1.70904e+01 \\ \hline
40 & 1.00401e-07 & 2.70230e+02 & 1.74852e+01 \\ \hline
41 & 6.78265e-08 & 5.55432e+02 & 3.25176e+01 \\ \hline
42 & 3.89531e-08 & 3.19532e+02 & 1.45562e+01 \\ \hline
43 & 7.07625e-08 & 3.53136e+02 & 1.21748e+01 \\ \hline
44 & 4.28090e-08 & 3.75632e+02 & 1.37852e+01 \\ \hline
45 & 4.61908e-08 & 4.01859e+02 & 9.50281e+00 \\ \hline
46 & 4.47556e-08 & 1.08220e+03 & 2.68951e+01 \\ \hline
47 & 1.06853e-07 & 4.67922e+03 & 4.85818e+01 \\ \hline
48 & 1.01608e-07 & 1.75233e+03 & 9.09009e+00 \\ \hline
49 & 5.73074e-07 & 5.92046e+03 & 1.69193e+02 \\ \hline
50 & 1.47107e-07 & 2.46086e+02 & 7.41959e+00 \\ \hline
51 & 9.47365e-08 & 8.57508e+02 & 3.73456e+01 \\ \hline
\end{tabular}
\hspace*{0cm}
}
\parbox{.45\linewidth}{
\hspace*{0cm}
\begin{tabular}{|l|l|l|l|}
\hline
$n$ & float128 & float64 & float32 \\ \hline
52 & 1.16113e-07 & 1.61072e+04 & 9.53346e+02 \\ \hline
53 & 8.12417e-08 & 2.58046e+02 & 1.80353e+01 \\ \hline
54 & 1.35778e-07 & 5.96323e+02 & 1.99145e+01 \\ \hline
55 & 9.55140e-08 & 5.90284e+02 & 1.11223e+01 \\ \hline
56 & 1.02527e-07 & 1.24163e+02 & 1.10942e+02 \\ \hline
57 & 1.31040e-07 & 1.01673e+02 & 1.37209e+01 \\ \hline
58 & 1.07265e-07 & 1.20849e+03 & 5.19018e+01 \\ \hline
59 & 3.79130e-08 & 9.46613e+02 & 4.18553e+01 \\ \hline
60 & 3.90234e-08 & 2.75634e+02 & 2.14181e+01 \\ \hline
61 & 9.38645e-08 & 2.68919e+02 & 6.71575e+01 \\ \hline
62 & 1.34922e-07 & 1.09973e+03 & 8.14264e+01 \\ \hline
63 & 8.84059e-07 & 4.24254e+03 & 3.31771e+02 \\ \hline
64 & 6.89870e-08 & 3.73795e+02 & 2.33192e+02 \\ \hline
65 & 6.60796e-08 & 3.79244e+02 & 8.67555e+01 \\ \hline
66 & 7.23018e-08 & 6.00618e+02 & 1.01882e+02 \\ \hline
67 & 1.03184e-07 & 6.39984e+02 & 5.68154e+01 \\ \hline
68 & 7.58521e-08 & 5.05802e+02 & 5.60500e+01 \\ \hline
69 & 7.60955e-08 & 1.28675e+03 & 1.43560e+02 \\ \hline
70 & 8.48296e-08 & 9.68115e+02 & 6.86839e+01 \\ \hline
71 & 7.66839e-08 & 2.09254e+02 & 3.09189e+01 \\ \hline
72 & 2.93301e-07 & 4.05753e+02 & 4.63669e+01 \\ \hline
73 & 8.10593e-08 & 4.11489e+02 & 3.42964e+01 \\ \hline
74 & 8.26313e-08 & 2.38284e+02 & 8.09815e+01 \\ \hline
75 & 8.04852e-08 & 7.10297e+02 & 1.62780e+02 \\ \hline
76 & 9.18682e-08 & 4.12513e+02 & 2.99406e+01 \\ \hline
77 & 1.08387e-07 & 1.87440e+02 & 2.79378e+01 \\ \hline
78 & 1.03110e-07 & 2.03857e+02 & 2.57488e+01 \\ \hline
79 & 1.37492e-07 & 6.78755e+02 & 3.58451e+01 \\ \hline
80 & 9.16911e-08 & 5.08628e+02 & 3.48783e+01 \\ \hline
81 & 5.00448e-07 & 6.06770e+02 & 4.10914e+01 \\ \hline
82 & 2.15135e-07 & 7.73650e+02 & 5.92223e+01 \\ \hline
83 & 8.16408e-07 & 1.17550e+03 & 1.05222e+02 \\ \hline
84 & 8.50474e-08 & 1.16940e+04 & 3.85955e+02 \\ \hline
85 & 1.25238e-06 & 1.03478e+04 & 1.20177e+02 \\ \hline
86 & 2.46212e-07 & 1.24309e+04 & 2.93697e+02 \\ \hline
87 & 1.34919e-07 & 1.50364e+04 & 1.45779e+02 \\ \hline
88 & 1.77220e-07 & 6.09119e+04 & 7.15888e+01 \\ \hline
89 & 2.10088e-07 & 1.41214e+03 & 1.00808e+02 \\ \hline
90 & 1.11207e-07 & 1.76003e+03 & 6.43347e+01 \\ \hline
91 & 1.40491e-07 & 9.43939e+02 & 4.96421e+01 \\ \hline
92 & 3.61188e-07 & 1.68947e+03 & 5.39511e+01 \\ \hline
93 & 7.47588e-07 & 1.23000e+03 & 3.33221e+01 \\ \hline
94 & 8.67451e-07 & 1.16441e+03 & 4.16179e+01 \\ \hline
95 & 1.40393e-06 & 1.11327e+03 & 4.31873e+01 \\ \hline
96 & 1.02786e-06 & 1.93032e+04 & 5.98646e+02 \\ \hline
97 & 1.94037e-07 & 9.62075e+02 & 3.08990e+01 \\ \hline
98 & 2.01763e-07 & 1.33434e+03 & 6.69423e+01 \\ \hline
99 & 9.39441e-06 & 1.52907e+03 & 9.06394e+01 \\ \hline
100 & 2.77768e-06 & 3.77706e+03 & 1.63594e+02 \\ \hline
\end{tabular}
\hspace*{-2.5cm}
}
\caption{Błędy}
\end{table}

Błędy powiększają się wraz ze wzrastającym rozmiarem układu jak rownież z malejącą prezycją użytego typu, przy czym należy
zauważyć, że błąd wynikający z precyzji jest znacznie wiekszy dla typów float32 oraz float 64 niż dla float128. Następnym krokiem
będzie porównanie wyników w problemie drugim z tymi uzyskanymi tutaj.

\section{Problem 2}
\subsection{Opis problemu}
Eksperyment analogiczny do problemu 1 zostanie przeprowadzony dla macierzy zadanej wzorem:
$$
\begin{cases}
    a_{1j}=\frac{2i}{j} \ dla \ j \ge i\\
    a_{ij}=a_{ij} \ dla \ j \le i
\end{cases}i,j=1,...,n
$$

Wyniki zostaną porównane z tymi w problemie 1. Zostanie sprawdzone uwarunkowanie obu układów.

\subsection{Opracowanie problemu}
Zakres badanych $n$, precyzji oraz uzyty wektor \textbf{x} są identyczne jak w problemie 1.
\begin{table}[H]
\centering
\begin{tabular}{|l|l|}
\hline
$n$ & Wynik (float64) \\ \hline
3 & [ 1. -1.  1.] \\ \hline
4 & [ 1. -1.  1. -1.] \\ \hline
5 & [ 1. -1.  1. -1.  1.] \\ \hline
6 & [ 1. -1.  1. -1.  1. -1.] \\ \hline
7 & [ 1. -1.  1. -1.  1. -1.  1.] \\ \hline
8 & [ 1. -1.  1. -1.  1. -1.  1. -1.] \\ \hline
9 & [ 1. -1.  1. -1.  1. -1.  1. -1.  1.] \\ \hline
10 & [ 1. -1.  1. -1.  1. -1.  1. -1.  1. -1.] \\ \hline
11 & [ 1. -1.  1. -1.  1. -1.  1. -1.  1. -1.  1.] \\ \hline
12 & [ 1. -1.  1. -1.  1. -1.  1. -1.  1. -1.  1. -1.] \\ \hline
13 & [ 1. -1.  1. -1.  1. -1.  1. -1.  1. -1.  1. -1.  1.] \\ \hline
14 & [ 1. -1.  1. -1.  1. -1.  1. -1.  1. -1.  1. -1.  1. -1.] \\ \hline
15 & [ 1. -1.  1. -1.  1. -1.  1. -1.  1. -1.  1. -1.  1. -1.  1.] \\ \hline
16 & [ 1. -1.  1. -1.  1. -1.  1. -1.  1. -1.  1. -1.  1. -1.  1. -1.] \\ \hline
17 & [ 1. -1.  1. -1.  1. -1.  1. -1.  1. -1.  1. -1.  1. -1.  1. -1.  1.] \\ \hline
18 & [ 1. -1.  1. -1.  1. -1.  1. -1.  1. -1.  1. -1.  1. -1.  1. -1.  1. -1.] \\ \hline
\end{tabular}
\caption{Wyniki dla wartości 3-18 $n$}
\end{table}

Wszystkie uzyskane wyniki w tym zakresie są na tyle dobre (tzn. mają na tyle dużo zer po przecinku),
że numpy nie pokazuje ich rozwinięcia dziesiętnego, w odróżneiniu od problemu 1, gdzie widać było ewidentne
niedokładności.

\newpage
\thispagestyle{empty}

\begin{table}[H]
\parbox{.45\linewidth}{
\hspace*{-2.4cm}
\begin{tabular}{|l|l|l|l|}
\hline
$n$ & float128 & float64 & float32 \\ \hline
3 & 0.00000e+00 & 3.14018e-16 & 3.14018e-16 \\ \hline
4 & 0.00000e+00 & 2.48253e-16 & 5.66105e-16 \\ \hline
5 & 2.48253e-16 & 4.15407e-16 & 4.96507e-16 \\ \hline
6 & 3.14018e-16 & 9.74217e-16 & 6.47366e-16 \\ \hline
7 & 2.48253e-16 & 1.69468e-15 & 7.77156e-16 \\ \hline
8 & 6.08094e-16 & 4.67218e-15 & 2.29416e-15 \\ \hline
9 & 1.48122e-15 & 3.31025e-15 & 1.42611e-15 \\ \hline
10 & 2.99143e-15 & 3.08274e-15 & 5.30473e-15 \\ \hline
11 & 2.98318e-15 & 4.42142e-15 & 6.66689e-15 \\ \hline
12 & 3.63842e-15 & 1.98040e-14 & 1.08353e-14 \\ \hline
13 & 4.76491e-15 & 2.20079e-14 & 9.17665e-15 \\ \hline
14 & 4.02599e-15 & 2.27677e-14 & 1.17500e-14 \\ \hline
15 & 3.61292e-15 & 2.83611e-14 & 1.40438e-14 \\ \hline
16 & 3.52485e-15 & 3.80102e-14 & 1.42723e-14 \\ \hline
17 & 1.00449e-14 & 3.71618e-14 & 1.41234e-14 \\ \hline
18 & 1.00792e-14 & 3.65198e-14 & 1.23634e-14 \\ \hline
19 & 1.52780e-14 & 3.88935e-14 & 1.32065e-14 \\ \hline
20 & 1.97189e-14 & 3.80900e-14 & 1.55973e-14 \\ \hline
21 & 2.00517e-14 & 3.65030e-14 & 1.69886e-14 \\ \hline
22 & 2.61846e-14 & 4.75961e-14 & 1.57937e-14 \\ \hline
23 & 2.92058e-14 & 4.25638e-14 & 1.89406e-14 \\ \hline
24 & 3.11027e-14 & 3.68189e-14 & 2.35490e-14 \\ \hline
25 & 4.01138e-14 & 3.90851e-14 & 2.60590e-14 \\ \hline
26 & 4.78317e-14 & 3.97499e-14 & 3.78837e-14 \\ \hline
27 & 4.94930e-14 & 4.30520e-14 & 4.32887e-14 \\ \hline
28 & 5.64670e-14 & 1.01729e-13 & 4.83767e-14 \\ \hline
29 & 6.79743e-14 & 1.22073e-13 & 6.59737e-14 \\ \hline
30 & 7.74665e-14 & 9.93568e-14 & 6.37943e-14 \\ \hline
31 & 7.20535e-14 & 1.23333e-13 & 8.47986e-14 \\ \hline
32 & 7.49745e-14 & 1.22503e-13 & 1.03770e-13 \\ \hline
33 & 8.34764e-14 & 1.20980e-13 & 1.00354e-13 \\ \hline
34 & 9.65992e-14 & 1.27744e-13 & 8.96049e-14 \\ \hline
35 & 1.03406e-13 & 1.25167e-13 & 8.98755e-14 \\ \hline
36 & 1.05284e-13 & 1.81115e-13 & 9.92374e-14 \\ \hline
37 & 9.74833e-14 & 1.78416e-13 & 8.58149e-14 \\ \hline
38 & 8.99382e-14 & 1.91015e-13 & 8.43823e-14 \\ \hline
39 & 9.39602e-14 & 1.75834e-13 & 8.63363e-14 \\ \hline
40 & 1.06354e-13 & 2.56295e-13 & 9.07217e-14 \\ \hline
41 & 9.88085e-14 & 2.55501e-13 & 8.50089e-14 \\ \hline
42 & 1.18753e-13 & 2.56017e-13 & 9.76102e-14 \\ \hline
43 & 1.23169e-13 & 2.59007e-13 & 9.94367e-14 \\ \hline
44 & 1.65062e-13 & 2.20919e-13 & 1.02333e-13 \\ \hline
45 & 1.80798e-13 & 2.21368e-13 & 1.19101e-13 \\ \hline
46 & 1.42598e-13 & 2.43778e-13 & 1.46323e-13 \\ \hline
47 & 1.37076e-13 & 3.02347e-13 & 1.48848e-13 \\ \hline
48 & 1.49469e-13 & 3.16843e-13 & 1.58636e-13 \\ \hline
49 & 1.38619e-13 & 3.23690e-13 & 1.72794e-13 \\ \hline
50 & 1.36274e-13 & 3.46057e-13 & 1.67102e-13 \\ \hline
51 & 1.43897e-13 & 3.58384e-13 & 1.85238e-13 \\ \hline
\end{tabular}
\hspace*{0cm}
}
\parbox{.45\linewidth}{
\hspace*{0cm}
\begin{tabular}{|l|l|l|l|}
\hline
$n$ & float128 & float64 & float32 \\ \hline
52 & 1.39772e-13 & 3.67188e-13 & 2.47585e-13 \\ \hline
53 & 1.65545e-13 & 3.71622e-13 & 2.44301e-13 \\ \hline
54 & 1.83550e-13 & 3.95670e-13 & 2.80490e-13 \\ \hline
55 & 1.50536e-13 & 4.47478e-13 & 3.02780e-13 \\ \hline
56 & 1.47575e-13 & 5.07994e-13 & 3.29484e-13 \\ \hline
57 & 1.63723e-13 & 5.18223e-13 & 3.51037e-13 \\ \hline
58 & 1.77244e-13 & 5.38007e-13 & 3.61192e-13 \\ \hline
59 & 1.88617e-13 & 5.32264e-13 & 3.53027e-13 \\ \hline
60 & 2.01146e-13 & 8.55487e-13 & 3.90212e-13 \\ \hline
61 & 1.81530e-13 & 8.64971e-13 & 3.94910e-13 \\ \hline
62 & 1.87015e-13 & 8.52088e-13 & 4.01676e-13 \\ \hline
63 & 1.92385e-13 & 8.53760e-13 & 4.09343e-13 \\ \hline
64 & 2.34847e-13 & 1.02174e-12 & 3.75456e-13 \\ \hline
65 & 2.29063e-13 & 1.08640e-12 & 5.19077e-13 \\ \hline
66 & 2.41997e-13 & 1.20444e-12 & 5.83299e-13 \\ \hline
67 & 2.62325e-13 & 1.27659e-12 & 6.21968e-13 \\ \hline
68 & 2.83034e-13 & 8.29059e-13 & 6.88447e-13 \\ \hline
69 & 2.75237e-13 & 8.79906e-13 & 7.48081e-13 \\ \hline
70 & 3.04656e-13 & 9.25090e-13 & 8.04045e-13 \\ \hline
71 & 3.06542e-13 & 9.76631e-13 & 8.66095e-13 \\ \hline
72 & 3.02185e-13 & 1.20634e-12 & 8.94100e-13 \\ \hline
73 & 3.29546e-13 & 1.20252e-12 & 9.29830e-13 \\ \hline
74 & 3.27333e-13 & 1.20104e-12 & 9.24748e-13 \\ \hline
75 & 3.19322e-13 & 1.20343e-12 & 9.47747e-13 \\ \hline
76 & 3.34962e-13 & 1.89808e-12 & 9.03211e-13 \\ \hline
77 & 3.73973e-13 & 1.88248e-12 & 8.68586e-13 \\ \hline
78 & 4.06460e-13 & 1.96682e-12 & 8.98199e-13 \\ \hline
79 & 4.22540e-13 & 1.96288e-12 & 9.04314e-13 \\ \hline
80 & 4.27885e-13 & 1.57121e-12 & 8.69675e-13 \\ \hline
81 & 5.00671e-13 & 1.66451e-12 & 9.31137e-13 \\ \hline
82 & 5.15757e-13 & 1.97529e-12 & 9.47249e-13 \\ \hline
83 & 5.69078e-13 & 2.05867e-12 & 9.50470e-13 \\ \hline
84 & 6.33495e-13 & 2.30233e-12 & 9.70425e-13 \\ \hline
85 & 6.50080e-13 & 2.28787e-12 & 1.01391e-12 \\ \hline
86 & 6.48275e-13 & 2.28022e-12 & 1.00284e-12 \\ \hline
87 & 7.06903e-13 & 2.29377e-12 & 1.09560e-12 \\ \hline
88 & 7.49646e-13 & 2.21554e-12 & 1.14357e-12 \\ \hline
89 & 8.26162e-13 & 2.24017e-12 & 1.16569e-12 \\ \hline
90 & 8.57090e-13 & 2.39984e-12 & 1.19415e-12 \\ \hline
91 & 8.54227e-13 & 2.69002e-12 & 1.24766e-12 \\ \hline
92 & 8.90434e-13 & 2.24959e-12 & 1.23540e-12 \\ \hline
93 & 9.01505e-13 & 2.26938e-12 & 1.27993e-12 \\ \hline
94 & 9.70675e-13 & 2.29189e-12 & 1.32937e-12 \\ \hline
95 & 9.36551e-13 & 2.27211e-12 & 1.41987e-12 \\ \hline
96 & 9.29838e-13 & 2.46800e-12 & 1.39482e-12 \\ \hline
97 & 9.26189e-13 & 2.46843e-12 & 1.42324e-12 \\ \hline
98 & 9.77432e-13 & 2.51404e-12 & 1.48752e-12 \\ \hline
99 & 1.16504e-12 & 2.54730e-12 & 1.53327e-12 \\ \hline
100 & 1.21042e-12 & 2.28769e-12 & 1.55604e-12 \\ \hline
\end{tabular}
\hspace*{-2.5cm}
}
\caption{Błędy}
\end{table}

Nawet dla największej badanej wartości $n$, niezależnie od precyzji użytego typu, pojawiający się błąd jest
rzędu $10^{-12}$. Precyzja uzyskanych wyników w problemie 2 w takich samych warunkach jest znacznie lepsza niż w 
problemie 1.

Tutaj należałoby się zastanowić nad powodem powyższej różnicy wyników pomiędzy macierzami w problemie pierwszym i drugim.
Poniżej części macierzy \textbf{A} dla $n=100$ dla obu problemów:

$$
A_1 = 
\begin{bmatrix}
1 & 1 & 1  &  \hdots & 1 & 1 & 1\\
0.5   &     0.33333333 & 0.25   &  \hdots & 0.01010101 & .01 & 0.00990099\\
0.33333333 & 0.25    &   0.2     &   \hdots & 0.01   &    0.00990099& 0.00980392\\
\vdots & & & \ddots & & & \vdots\\
0.01020408 & 0.01010101 & 0.01  &     \hdots & 0.00512821 & 0.00510204  & 0.00507614\\
0.01010101 & 0.01  &     0.00990099 & \hdots & 0.00510204 & 0.00507614 & 0.00505051\\
0.01   &    0.00990099 & 0.00980392 & \hdots & 0.00507614 & 0.00505051 & 0.00502513\\
\end{bmatrix}
$$

$$
A_2 = 
\begin{bmatrix}
2  &       1.    &     0.66666667 & \hdots & 0.02040816 & 0.02020202 & 0.02\\ 
1.    &     2.     &    1.33333333 & \hdots & 0.04081633 & 0.04040404 & 0.04\\
0.66666667 & 1.33333333 & 2.    &     \hdots 0.06122449 & 0.06060606 & 0.06\\
\vdots & & & \ddots & & & \vdots\\
0.02040816 & 0.04081633 & 0.06122449 & \hdots & 2.     &    1.97979798 & 1.96\\
0.02020202 & 0.04040404 & 0.06060606 & \hdots & 1.97979798 & 2.         & 1.98\\
0.02     &  0.04    &   0.06    &  \hdots & 1.96    &   1.98   &    2.\\
\end{bmatrix}
$$

Jedną z przyczyn, dla których układ równań z problemu pierwszego daje tak złe wyniki jest metoda wybierania elementu wiodącego
(pivotu). W tej implementacji algorytmu eliminacja Gaussa jako pivot wybierane są zawsze kolejne elementy na przekątnej 
macierzy. W przypadku macierzy z problemu drugiego jest to zawsze 2, w przypadku macierzy z problemu pierwszego elementy
na przekątnej maleją, tak że najmniejsze wartości są rzędu $10^{-3}$. Może to stanowić problem, ponieważ poszczególne wiersze
w metodzie eliminacji Gaussa są dzielone przez pivot, to znaczy mnożone przez jego odwrotność. Jeżeli pivot jest mały ($<1$), wówczas
wiersze mnożone są przez dużą wartość, więc błędy zaokrągleń stają się duże w stosunku do współczynników oryginalnej macierzy.

Oprócz tego można również obliczyć wskaźnik uwarunkowania macierzy $\kappa$. Wartość ta jest miarą tego, jak bardzo zmieni się rozwiązanie
\textbf{x} układu równań w stosunku do zmiany \textbf{b}. Jeżeli wskaźnik uwarunkowania macierzy jest duży, nawet mały błąd w 
\textbf{b} może powodować duże błędy w \textbf{x}.

$$\kappa=\|A^{-1}\|\|A\|$$

Można zastosować dowolną normę zawartą, tutaj zostanie użyta norma ``nieskończoność'', tj. $\displaystyle\|A\|_{\infty}=\max_{i}\sum_{j}|a_{ij}|$.

Poniżej tabela z wybranymi wskaźnikami uwarunkowania macierzy dla różnych $n$:

\begin{table}[H]
\centering
\begin{tabular}{|l|l|l|}
\hline
$n$ & $\kappa_1$ (problem 1) & $\kappa_2$ (problem 2)\\ \hline
3 & 8.64000e+02 & 8.66667e+00 \\ \hline
6 & 5.63447e+07 & 3.96667e+01 \\ \hline
9 & 2.84317e+12 & 9.23810e+01 \\ \hline
12 & 1.36454e+17 & 1.66822e+02 \\ \hline
15 & 1.04335e+19 & 2.63294e+02 \\ \hline
18 & 2.21394e+21 & 3.81736e+02 \\ \hline
21 & 1.89813e+19 & 5.21917e+02 \\ \hline
24 & 1.51520e+19 & 6.83833e+02 \\ \hline
27 & 2.25108e+20 & 8.67483e+02 \\ \hline
30 & 5.08850e+19 & 1.07287e+03 \\ \hline
33 & 6.34109e+19 & 1.30055e+03 \\ \hline
36 & 3.60448e+20 & 1.54997e+03 \\ \hline
39 & 2.50037e+20 & 1.82111e+03 \\ \hline
42 & 6.68682e+19 & 2.11399e+03 \\ \hline
45 & 3.72911e+21 & 2.42860e+03 \\ \hline
48 & 3.75331e+20 & 2.76521e+03 \\ \hline
51 & 1.67838e+21 & 3.12386e+03 \\ \hline
54 & 4.26339e+20 & 3.50424e+03 \\ \hline
57 & 6.19806e+20 & 3.90635e+03 \\ \hline
60 & 9.06089e+20 & 4.33019e+03 \\ \hline
63 & 2.47455e+20 & 4.77576e+03 \\ \hline
66 & 2.37977e+20 & 5.24360e+03 \\ \hline
69 & 1.79781e+21 & 5.73321e+03 \\ \hline
72 & 7.20453e+20 & 6.24455e+03 \\ \hline
75 & 5.53474e+21 & 6.77762e+03 \\ \hline
78 & 1.18038e+21 & 7.33242e+03 \\ \hline
81 & 8.58646e+21 & 7.90918e+03 \\ \hline
84 & 4.81463e+20 & 8.50802e+03 \\ \hline
87 & 1.01666e+21 & 9.12860e+03 \\ \hline
90 & 9.19833e+20 & 9.77090e+03 \\ \hline
93 & 5.60807e+20 & 1.04349e+04 \\ \hline
96 & 1.08158e+21 & 1.11207e+04 \\ \hline
99 & 2.33922e+21 & 1.18287e+04 \\ \hline
\end{tabular}
\caption{wskaźniki uwarunkowania macierzy}
\end{table}

Wskaźniki uwarunkowania dla macierzy z problemu 1 są znacznie więsze niż dla problemu 2.
Oznacza to, że niewielki błąd (wynikający np. z przybliżeń) znacznie wpływa na wynik.

\section{Problem 3}
Ekspetyment z dwóch poprzednich problemów zostanie powtórzony dla macierzy zadanej wzorem:
$$
\begin{cases}
    a_{ii}=k\\
    a_{i,i+1}=\frac{1}{i+m}\\
    a_{i,i-1}=\frac{k}{i+m+1} \ dla \ i > j\\
    a_{i,j}=0 \ dla \ j<i-1 \ oraz \ j>i+1
\end{cases}i,j=1,...,n
$$
Gdzie $k=6$ oraz $m=5$.

Następnie układ rozwiązany zostanie metodą przeznaczoną do rozwiązywania układów z macierzą trójdiagonalną. Te dwie metody
zostaną porównane (czas, dokładność obliczeń i zajętość pamięci) dla różnych rozmiarów układu (z pominięciem czasu tworzenia
układu). Opisane zostanie to, jak w metodzie dla ukłądów z macierzą trójdiagonalną przechowywano i wykorzystano macierz \textbf{A}.

\subsection{Opracowanie problemu}

\subsection{Wnioski}

\end{document}