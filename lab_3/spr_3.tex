\documentclass{article}

\usepackage{polski}
\usepackage{amsmath}
\usepackage{graphicx}
\usepackage{float}
\usepackage{subfig}
\usepackage{multirow}

\title{Interpolacja funkcjami sklejanymi}
\author{\textbf{Łukasz Wala}\\
    \textit{AGH, Wydział Informatyki, Elektroniki i Telekomunikacji} \\
    \textit{Metody Obliczeniowe w Nauce i Technice 2021/2022}}
\date{Kraków, \today}

\begin{document}
\maketitle

\section{Opis problemu}
Główną ideą zadania jest zbadanie zachowania wielomianów interpolacyjnych
dla poniższej funkcji skonstruowanych za pomocą funkcji sklejanych drugiego oraz
trzeciego stopnia dla równych warunków brzegowych.

Badana funkcja:
\[f(x)=x^2-m\cdot\cos\left(\frac{\pi x}{k}\right)\]
Gdzie $k=\frac{1}{2}$, $m=4$ oraz $x\in [-6,6]$.


\end{document}