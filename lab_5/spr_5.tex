\documentclass{article}

\usepackage{polski}
\usepackage{amsmath}
\usepackage{graphicx}
\usepackage{float}
\usepackage{subfig}
\usepackage{multirow}

\title{Rozwiązywanie równań i układów równań nieliniowych}
\author{\textbf{Łukasz Wala}\\
    \textit{AGH, Wydział Informatyki, Elektroniki i Telekomunikacji} \\
    \textit{Metody Obliczeniowe w Nauce i Technice 2021/2022}}
\date{Kraków, \today}

\begin{document}
\maketitle

\section{Problem 1}
\subsection{Opis problemu}
Główną ideą zadanie jest wyznaczenie pierwiastków równania f(x)=0 w zadanym przedziale.
Dla metody Newtona punkty wybierane będą rozpoczynając od wartości końców przedziału, zmniejszając je o 0.1 w kolejnych eksperymentach numerycznych.
Odpowiednio dla metody siecznej jeden z końców przedziału stanowić powinna wartość punktu startowego dla 
metody Newtona, a drugi - początek, a następnie koniec przedziału [a, b].

Badana funkcja:
\[f(x)=mxe^{-n}-me^{-nx}+1/m\]
Gdzie $n=9$, $m=25$ oraz $x\in [0.1,1.9]$.

\subsection{Opracowanie}
\subsection{Wnioski}

\section{Problem 2}
Główną ideą problemu jest rozwiązanie układu równań metodą Newtona.
\[
\begin{cases}
    x^2_1+x^2_2-x^2_3=1 \\
    x_1-2x^3_2+2^2_3=-1 \\
    2x_1^2+x_2-2x^2_3=1
\end{cases}
\]

Eksperymenty zostaną przeprowadzone dla różnych wektorów początkowych. 
Sprawdzona zostanie liczba rozwiązań układu, przy jakich wektorach początkowych metoda 
nie zbiega do rozwiązania oraz to jakie wektory początkowe doprowadzają do jakiego 
rozwiązania. Zastosowane zostaną dwa różne kryteria stopu.

\subsection{Opis problemu}
\subsection{Opracowanie}
\subsection{Wnioski}

\end{document}