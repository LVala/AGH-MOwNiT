\documentclass{article}

\usepackage{polski}
\usepackage{amsmath}
\usepackage{graphicx}
\usepackage{float}
\usepackage{subfig}
\usepackage{multirow}

\title{Rozwiązywanie układów równań liniowych metodami iteracyjnymi}
\author{\textbf{Łukasz Wala}\\
    \textit{AGH, Wydział Informatyki, Elektroniki i Telekomunikacji} \\
    \textit{Metody Obliczeniowe w Nauce i Technice 2021/2022}}
\date{Kraków, \today}

\begin{document}
\maketitle

\section{Problem 1}
\subsection{Opis problemu}
Dany jest układ równań liniowych \textbf{Ax}=\textbf{b}.
Elementy macierzy \textbf{A} o wymiarze $n$\,x\,$n$ są określone wzorem:
$$
\begin{cases}
    a_{i,i}=k\\
    a_{i,j}=\frac{1}{|i-j|+m} \ dla \ i \ne j
\end{cases}i,j=1,...,n
$$

Gdzie $k=8$, $m=3$.

Układ zostanie rozwiązany metodą Jakobiego. Obliczenia zostaną wykonane dla różnych $n$,
dla różnych wektorów początkowych oraz różnych wartości $\rho$ w kryteriach stopu. Wyznaczone zostaną: liczba iteracji,
różnica w czasie obliczeń dla obu kryteriów stopu. Sprawdzona zostanie dokładność obliczeń.

Użyte kryteria stopu (norma euklidesowa):
\begin{enumerate}
    \item 
    $\left\|x^{(i+1)}-x^{(i)}\right\| < \rho$
    \item
    $\left\|Ax^{(i)}-b\right\| < \rho$
\end{enumerate}

\subsection{Opracowanie problemu}
Program użyty do rozwiązania układu został napisany w języku Python z użyciem pakietu numpy.

\subsection{Wnioski}




\section{Problem 2}
\subsection{Opis problemu}
Przy użyciu dowolnej metody zostanie znaleziony promień spektralny macierzy iteracji z poprzedniego problemu
(dla różnych rozmiarów układu --- takich, dla których znajdowane były rozwiązania układu).
Sprawdzone zostanie, czy spełnione są założenia o zbieżności metody dla zadanego układu. Opisana zostanie metoda znajdowania
promienia spektralnego.

\subsection{Opracowanie problemu}

\subsection{Wnioski}


\end{document}