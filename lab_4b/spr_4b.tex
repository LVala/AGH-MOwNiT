\documentclass{article}

\usepackage{polski}
\usepackage{amsmath}
\usepackage{graphicx}
\usepackage{float}
\usepackage{subfig}
\usepackage{multirow}

\title{Aproksymacja średniokwadratowa wielomianami trygonometrycznymi}
\author{\textbf{Łukasz Wala}\\
    \textit{AGH, Wydział Informatyki, Elektroniki i Telekomunikacji} \\
    \textit{Metody Obliczeniowe w Nauce i Technice 2021/2022}}
\date{Kraków, \today}

\begin{document}
\maketitle

\section{Opis problemu}
Główną ideą zadania jest zbadanie zachowania funkcji przybliżonej za pomocą aproksymacji
średniokwadratowej wielomianami trygonometrycznymi.

Badana funkcja:
\[f(x)=x^2-m\cdot\cos\left(\frac{\pi x}{k}\right)\]
Gdzie $k=\frac{1}{2}$, $m=4$ oraz $x\in [-6,6]$.

\section{Opracowanie}
\subsection{Wyprowadzenie}
W aproksymacji średniokwadratowej poszukiwana jest wartość minimalna sumy kwadratów różnic funkcji aproksymowanej $F(x)$
oraz funkcji aproksymującej $f(x)$ z uwzględnieniem funcji wagowej $w(x)$ większej od zera (tutaj $\forall x \in D
:w(x) = 1$). Funkcją aproksymującą ma być wielomian trygonometryczny o postaci
$$\displaystyle f(x) = a_0 + \sum_{j=1}^mb_j\cos(jx)\:+\:\sum_{j=1}^m a_j\sin(jx)$$

Więc błąd średniokwadratowy przyjmuje postać

$$\displaystyle H(a_0, ..., a_m, b_1, ..., b_m)=\sum_{i=1}^nw(x_i)\left[F(x_i)-(a_0 + \sum_{j=1}^mb_j\cos(jx_i)\:+\:\sum_{j=1}^m a_j\sin(jx_i))\right]^2$$

Aby funkcja przyjmowała wartość minimalną względem współczynnka $c$, pochodna funkcji względem tego współczynnika musi
wynosić zero

$$\frac{\partial H}{\partial c}=0, c \in \{a_0, ..., a_m, b_1, ..., b_m\}$$

Na przykład dla $a_k$

$$\displaystyle -2\sum_{i=1}^nw(x_i)\left[F(x_i)-(a_0 + \sum_{j=1}^mb_j\sin(jx_i)\:+\:\sum_{j=1}^m a_j\cos(jx_i))\right]\cos(kx_i) = 0$$

Po przekształceniach dla $a_0$
\begin{multline*}
    $$\displaystyle \sum_{i=0}^nw(x_i)a_0+\sum_{j=1}^m\left(\sum_{i=0}^nw(x_i)\cos(jx_i)\right)a_j+
    \sum_{j=1}^m\left(\sum_{i=0}^nw(x_i)\sin(jx_i)\right)b_j \\=\sum_{i=0}^nw(x_i)F(x_i)$$
\end{multline*}

Dla $a_k, k\in\{1,2,...,m\}$:
\begin{multline*}
    $$\displaystyle \sum_{i=0}^nw(x_i)\cos(kx_i)a_0+\sum_{j=1}^m\left(\sum_{i=0}^nw(x_i)\cos(kx_i)\cos(jx_i)\right)a_j+
    \sum_{j=1}^m\left(\sum_{i=0}^nw(x_i)\cos(kx_i)\sin(jx_i)\right)b_j \\ =\sum_{i=0}^nw(x_i)F(x_i)\cos(kx_i)$$
\end{multline*}

Oraz dla $b_k, k\in\{1,2,...,m\}$
\begin{multline*}
    $$\displaystyle \sum_{i=0}^nw(x_i)\sin(kx_i)a_0+\sum_{j=1}^m\left(\sum_{i=0}^nw(x_i)\sin(kx_i)\cos(jx_i)\right)a_j+
    \sum_{j=1}^m\left(\sum_{i=0}^nw(x_i)\sin(kx_i)\sin(jx_i)\right)b_j \\ =\sum_{i=0}^nw(x_i)F(x_i)\sin(kx_i)$$
\end{multline*}

Z tych równań można zbudować układ
\[
\begin{bmatrix}
    \sum_{i=0}^nw(x_i) & \sum_{i=0}^nw(x_i)\cos(1\cdot x_i) & \sum_{i=0}^nw(x_i)\sin(1\cdot x_i) & \hdots \\
    \sum_{i=0}^nw(x_i)\cos(1\cdot x_i) & \sum_{i=0}^nw(x_i)\cos(1\cdot x_i)\cos(1\cdot x_i) & \sum_{i=0}^nw(x_i)\cos(1\cdot x_i)\sin(1\cdot x_i) & \hdots\\
    \sum_{i=0}^nw(x_i)\sin(1\cdot x_i) & \sum_{i=0}^nw(x_i)\sin(1\cdot x_i)\cos(1\cdot x_i) & \sum_{i=0}^nw(x_i)\sin(1\cdot x_i)\sin(1\cdot x_i) & \hdots\\
    \sum_{i=0}^nw(x_i)\cos(2\cdot x_i) & \sum_{i=0}^nw(x_i)\cos(2\cdot x_i)\cos(1\cdot x_i) & \sum_{i=0}^nw(x_i)\cos(2\cdot x_i)\sin(1\cdot x_i) & \hdots\\
    \sum_{i=0}^nw(x_i)\sin(2\cdot x_i) & \sum_{i=0}^nw(x_i)\sin(2\cdot x_i)\cos(1\cdot x_i) & \sum_{i=0}^nw(x_i)\sin(2\cdot x_i)\sin(1\cdot x_i) & \hdots\\
    \vdots & \vdots & \vdots & \vdots \\
    \sum_{i=0}^nw(x_i)\cos(m\cdot x_i) & \sum_{i=0}^nw(x_i)\cos(m\cdot x_i)\cos(1\cdot x_i) & \sum_{i=0}^nw(x_i)\cos(m\cdot x_i)\sin(1\cdot x_i) & \hdots\\
    \sum_{i=0}^nw(x_i)\sin(m\cdot x_i) & \sum_{i=0}^nw(x_i)\sin(m\cdot x_i)\cos(1\cdot x_i) & \sum_{i=0}^nw(x_i)\sin(m\cdot x_i)\sin(1\cdot x_i) & \hdots\\
\end{bmatrix}
\cdot
\]

\[
\cdot
\begin{bmatrix}
    a_0 \\
    a_1 \\
    b_1 \\
    a_2 \\
    b_2 \\
    \vdots \\
    a_m \\
    b_m \\ 
\end{bmatrix}
=
\begin{bmatrix}
    \sum_{i=0}^nw(x_i)F(x_i) \\
    \sum_{i=0}^nw(x_i)F(x_i)\cos(1\cdot x_i) \\
    \sum_{i=0}^nw(x_i)F(x_i)\sin(1\cdot x_i) \\
    \sum_{i=0}^nw(x_i)F(x_i)\cos(2\cdot x_i) \\
    \sum_{i=0}^nw(x_i)F(x_i)\sin(2\cdot x_i) \\
    \vdots \\
    \sum_{i=0}^nw(x_i)F(x_i)\cos(m\cdot x_i) \\ 
    \sum_{i=0}^nw(x_i)F(x_i)\sin(m\cdot x_i)
\end{bmatrix}    
\]

Do stworzenia i rozwiązania powyższego układu równań użyta została funkcja \textit{linalg.solve} z pakietu \textit{numpy}
w języku Python.

\subsection{Wykresy}



\subsection{Dokładności}

\section{Wnioski}


\end{document}